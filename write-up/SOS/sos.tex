\documentclass{article}
\usepackage[]{amsmath}
\usepackage{amsthm} 
\usepackage{amsbsy}
% \newtheorem*{remark}{Main Result}
\usepackage{breqn}
\usepackage{graphics} % for pdf, bitmapped graphics files
\usepackage{epsfig} % for postscript graphics files
\usepackage{amssymb}  % assumes amsmath package installed
\usepackage{graphicx}
\usepackage{caption}
\usepackage{subcaption}
\usepackage{multirow}
\usepackage[margin=1in]{geometry}
\newcommand{\overbar}[1]{\mkern 1.5mu\overline{\mkern-1.5mu#1\mkern-1.5mu}\mkern 1.5mu}
\newtheorem*{remark}{Remark}
% \usepackage{todonotes}
% \usepackage{algorithmic}
% \usepackage[numbered, framed]{mcode}
% \newtheorem{remark}{Remark}
\begin{document}




\section{Composable Verification for Polynomial System} % (fold)
\label{sec:composable_verification_for_polynomial_system}
Consider a polynomial system $\dot x=f(x)$ where the state vector $x\in \mathbb{R}^m$. Partition $x$ into $n$ disjoint parts $x=[x_1;x_2;\dots;x_n]$ where $x_i \in \mathbb{R}^{m_i}$ and $\sum\limits_{i}{m_i}=m$, and write the dynamics of the sub-systems as $\dot{x_i}=f_i(x_i)+g_i(x)$, $i=1,2,\dots,n$, that is, $f_i$ is a polynomial function purely of $x_i$, whereas $g_i$ is a function of the original system states $x$. For simplicity, group any constant terms into $f_i$, so then $g_i$ consists of all the cross-product terms.

$\lnot$


If 
then $V=\sum\limits_{i}{V_i}$ is a Lyapunov function for the original system, i.e. $\frac{\partial{V}}{x}<0$

$\frac{\partial{V}}{x}=\sum\limits_{i}{\frac{\partial{V_i}}{x_i}\dot{x_i}}=\sum\limits_{i}{\frac{\partial{V_i}}{x_i}(f_i(x_i)+g_i(x)})$


\section{Polynomial extension to Schur complement} % (fold)
\label{sec:polynomial_extension_to_schur_complement}
Consider the following problem: Given a polynomial function $p(x)$, which is known to be sum-of-squares, find a polynomial function $f(x)$ of a fixed degree such that $p(x)-f(x)^2$ is SOS. At first glance, due to the quadratic term $f(x)^2$, the constraints involves that is quadratic in the decision variables, illegal in the low-level SDP solvers' problem setup. However, 


The interpretation of the result above is immendiate from 



% section polynomial_extension_to_schur_complement (end)


\end{document}


